\documentclass[UTF8,twocolumn,fancyhdr]{ctexart}
\title{关于在初中数学竞赛操作题中的不变量的求解方法及其用途的探究}
\author{Moyan Liang,张江集团学校奥数班成员}
\date{\today}
\usepackage{amsthm,amsfonts,siunitx}
\usepackage{etoolbox}
\usepackage{asymptote}
\usepackage{newtxtext,newtxmath}
\usepackage{abstract}
\usepackage{cite}
\usepackage[colorlinks,linkcolor=blue]{hyperref}
\renewcommand{\abstractnamefont}{\normalfont\bfseries}
\renewcommand{\abstractnamefont}{\normalfont}
\makeatletter

\renewcommand*\maketitle{%
	\thispagestyle{empty}
	\let\footnotesize\small
	%\let\footnoterule\hrule\vspace{5pt}
    \let \footnote \thanks
    
	\begin{center}
        {\LARGE \@title \par}
        \vspace{2pt}
        {\Large \@author \par}
    \end{center}
    
	\@thanks
	\vfil\null
	\setcounter{footnote}{0}%
	\cleardoublepage
 \global\let\thanks\relax
 \global\let\maketitle\relax
 \global\let\@thanks\@empty
 \global\let\@author\@empty
 \global\let\@date\@empty
 \global\let\@title\@empty
 \global\let\title\relax
 \global\let\author\relax
 \global\let\date\relax
 \global\let\and\relax

}
\ctexset{
    section = {
        number=\arabic{section},
        name=\S,
        format+=\zihao{-4}
    },
    subsection = {
        number=\roman{subsection},
        name={(,).},
        format+=\zihao{-4}\textit
    }
}
\theoremstyle{plain}% default
	\newtheorem{theorem}{定理}[section] %
	\newtheorem{lemma}[theorem]{引理} %
	\newtheorem{proposition}[theorem]{命题} %
	\newtheorem*{corollary}{推论} %
	\theoremstyle{definition} %
	\newtheorem{definition}{定义}[section] %
	\newtheorem{conjecture}{猜想}[section] %
    \newtheorem{example}{例}[section] %
    \newtheorem{problem}{题}[section]
    \newtheorem*{problema}{引题}
	\theoremstyle{remark} %
	\newtheorem*{remark}{\normalfont\bfseries 评论} %
	\newtheorem*{note}{\normalfont\bfseries 注} %
    \newtheorem{case}{\normalfont\bfseries 案例} %
    \newtheorem*{solution}{解}
	\renewcommand*{\proofname}{\normalfont\bfseries\color{black}证明} %
    \renewcommand*{\qed}{\begin{flushright}
        $\square$
    \end{flushright}}
    
\makeatother

\begin{document}
    
    \twocolumn[\maketitle
    \begin{onecolabstract}
        在初中数学竞赛中常常会出现一种操作题,如移动硬币,分发糖果等.他们皆是先描述某些日常物品的移动,
        而最后问由一种物品状态到另一种状态的转移的可能性.
        本文主要探究了在初中数学竞赛操作题中的变量与"不变量"之间的有趣的关系,
        同时通过一些经典及崭新的问题,让读者了解在初中数学操作题中
        不变量的求解及使用方法,并分享自己在解此类问题的心得,并抒发自己的愚见.
    \end{onecolabstract}
    ]
    \section{引言}
    在数月前,我为了解答USAJMO-2019的第一题而询问罗家亮老师,老师给予了回答.
    \begin{problema}
	有$a+b$个碗排成一行,编号为$1$至$a+b$,$a$\text{与}$b$为给定正整数.最初的时候,前$a$个碗各包含一个苹果,后$b$个碗各包含一个梨.一个合法的移动包含将一个苹果从碗$i$移动到碗$i+1$,并将一个梨从碗$j$移动到碗$i-1$,只要差$i-j$是偶数.我们的目标是:将最初的梨和苹果的顺序颠倒位置(即,前$b$个碗各包含一个梨,后$a$个碗各包含一个苹果.)证明这在且仅在乘积$ab$是偶数时是可能的.
    \flushright{(Translator:Moyan Liang)}
		\label{pbl:ini}
    \end{problema}
    在这道题的证明中充分运用了"不变量"的思想,从而完成了本题的证明.证明将在\S\ref{sect2}中给出\footnote{值得一提的是:此题的完整解答须包含证明与构造可行的移动规则两个部分.笔者只完成了构造部分而被老师称为"本末倒置".}.
    本文共分三节,第一节介绍了一系列基础
        记法及定义,以及此文引题.第二节通过一系列问题引入"不变量"这一
        概念,同时介绍"不变量"的求解方法.第三节介绍"不变量"方法在现代
        数学竞赛中的应用与本人在研究此课题时的愚见.
    
    \section{"不变量"的求解方法}
下面给出引题的证明.
\begin{proof}[引题的证明]\footnote{此证明出自AOPS,由罗老师整理及翻译.}\footnote{这证明不包括构造过程,请参考本文附录.}
记奇数号碗中苹果和梨的总数分别为$A_1,P_1$,偶数号碗的苹果和梨的总数分别为$A_2,P_2$.
由于$(i-j)$时偶数,故每次操作都是将奇(偶)数号碗中的一个梨和苹果换到偶(奇)数号碗中,于是$(A_1-A_2)$
与$(P_1-P_2)$同时增加$2$或$-2$,这表明\begin{equation}
    M=(A_1-A_2)-(P_1-P_2)\label{eqn:inv}
\end{equation}
是本题合法操作的不变量.若$ab$为奇数,则$a,b$皆为奇数,初始状态下由\eqref{eqn:inv}
得$M=1-(-1)=2$,结束时$M=-1-1=-2$,由$M$为不变量推出矛盾,命题获证.
\end{proof}

此题的证明充分的利用了不变量的性质(就是说,它是不会变的),从而推出矛盾.事实上,在代数问题中
不变量亦是极其有用的.
\begin{problem}\footnote{出自:奥数教程(八年级),\S{8},例5.有改动.}
    已知三个数89,12,3,进行一种运算$Q$:\label{pbl:qar}
    \begin{equation}
        Q(a,b,c)=(\dfrac{a+b}{\sqrt{2}},\dfrac{a-b}{\sqrt{2}},c^2.)\label{eqn:qar}
    \end{equation}
    问:能否经过若干次运算$Q$,得到3个数90,14,10?证明它.
\end{problem}
此题初看没有头绪,而此类问题的一般解法是从\eqref{eqn:qar}中寻找不变量.
\begin{proof}
    \kaishu{答案显然是否定的.接下来给出证明.}
    \songti 我们知道
    \begin{equation}
        \left(\dfrac{a+b}{\sqrt{2}}\right)^2+\left(\dfrac{a-b}{\sqrt{2}}\right)^2+c^2=a^2+b^2+c^2.\label{eqn:eqv}
    \end{equation}
    \eqref{eqn:eqv}就是说,经过一次运算$Q$之前与之后的三个数的平方和是不变的.至此,我们找到了此题中的"不变量":平方和.
    而$${89}^2+12^2+3^2=8074,$$
    $$90^2+14^2+10^2=8396>8074.$$
    故经过题中题中规定的运算不能由已知的三个数得到$90,14,10$三个数.
\end{proof}
从上两例中我们可以看出"不变量"实际上是一种思想,要找到并证明它需要的是各种数学方法,如判别式,恒等变形等.下面分几类详细讨论.
\subsection{一般不变量}
\begin{problem}\footnote{选自罗老师提供的资料.}
    在某部落的语言中一共只有两个字母$A$与$B$,并且该语言具有以下性质:
    如果从单词中删去相邻的字母串$AB$,词义保持不变.或者说:单词中添加字母串$BA$
    或$AABB$,词义保持不变.

    问:$ABB=BAA$吗?
\end{problem}
\begin{solution}
    记每个单词为$(a,b)$,其中,$a$为该单词中$A$的个数,$b$为该单词中$B$的个数.
    现寻找"不变量".显然,在一串"保义变换"中的$a-b$都是不会变的.而$$ABB:(1,2),1-2=-1;\quad BAA:(2,1),2-1=1.$$
    故$$ABB\neq BAA.$$
\end{solution}
在此题中由于原本的条件是不足以使用的,因为字母数太少了,故考虑寻找"不变量".

在寻找不变量的过程中,我们需要在变量之中看出规律来.如题\ref{pbl:qar},是从无理式(准确的来说,是根式)中寻找规律;
在此类题目中,我们需要求运算前的三个数与运算后的三个数的$n$\footnote{此$n$是根式的次数.例如,在题\ref{pbl:qar}中,就是通过求其平方和.}次方和的关系.
\subsection{多项式中的"不变量"}
在下一题中,我们会发现纯粹的寻找不变量的方法是无用的;我们必须通过寻找判别式中的规律
中的不变量来寻找答案.
\begin{problem}
    对于二次三项式$ax^2+bx+c$,允许做下面的运算:
    \begin{enumerate}
        \item 将$a$与$c$对换;
        \item 把$x$换成$(x+t)$,其中,$t$为任意实数.重复作这样的运算,能把$x^2-x-2$换成$x^2-x-1$吗?
    \end{enumerate}
    重复作这样的运算,能把$x^2-x-2$换成$x^2-x-1$吗?
\end{problem}
\begin{solution}
    我们考虑判别式$\Delta$.第一种运算显然不改变$\Delta$.第二种运算不改变多项式两根之差.现有$$\Delta=b^2-4ac=a^2\left[\left(\dfrac{b}{a}\right)^2-4\dfrac{c}{a}\right]$$,而$-\dfrac{b}{a}=x
    _1+x_2$,$\dfrac{c}{a}=x_1x_2$,从而$\delta=a^2\left( x_1-x_2 \right)^2.$即第二种运算不改变$\delta.$而两个二次三项式的判别式为$9$与$5$,不能达到.
\end{solution}
\section{奇偶不变量与余数不变量}
在一些情况下,我们不能找到恒为定值的量,但我们可以考虑模算术.常见的模有:$2$(奇偶性),$4$(平方数的判断)等.下一道题中就是通过$\pmod 2$找出不变量,并解决问题.
\begin{problem}
    10名乒乓球运动员参加循环赛,每两名运动员之间都要进行比赛.在循环赛过程中,1号运动员获胜$x_1$次,失败$y_1$次;2号运动员获胜$x_2$次,失败$y_2$次,等等.求证:
$$x_1^2+x_2^2+…+x_{10}^2=y_1^2+y_{2}^2+…+y_{10}^2.$$

\end{problem}
    \bibliographystyle{plain}
    \nocite{*}
    \bibliography{bib/ref}

\end{document}