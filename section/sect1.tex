\section{引言}
    在数月前,我为了解答USAJMO-2019的第一题而询问罗家亮老师,老师给予了回答.
    \begin{problema}
	有$a+b$个碗排成一行,编号为$1$至$a+b$,$a$\text{与}$b$为给定正整数.最初的时候,前$a$个碗各包含一个苹果,后$b$个碗各包含一个梨.一个合法的移动包含将一个苹果从碗$i$移动到碗$i+1$,并将一个梨从碗$j$移动到碗$i-1$,只要差$i-j$是偶数.我们的目标是:将最初的梨和苹果的顺序颠倒位置(即,前$b$个碗各包含一个梨,后$a$个碗各包含一个苹果.)证明这在且仅在乘积$ab$是偶数时是可能的.
    \flushright{(Translator:Moyan Liang)}
		\label{pbl:ini}
    \end{problema}
    在这道题的证明中充分运用了"不变量"的思想,从而完成了本题的证明.证明将在\S\ref{sect2}中给出\footnote{值得一提的是:此题的完整解答须包含证明与构造可行的移动规则两个部分.笔者只完成了构造部分而被老师称为"本末倒置".}.
    本文共分三节,第一节介绍了一系列基础
        记法及定义,以及此文引题.第二节通过一系列问题引入"不变量"这一
        概念,同时介绍"不变量"的求解方法.第三节介绍"不变量"方法在现代
        数学竞赛中的应用与本人在研究此课题时的愚见.
    