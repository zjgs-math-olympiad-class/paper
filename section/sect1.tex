\section{引言}
    在数月前,我为了解答USAJMO-2019的第一题而询问罗家亮老师,老师给予了回答.
    \begin{problema}
        There are $a+b$ bowls arranged in a row, numbered $1$ through $a+b$ , where $a$ and  $b$ are given positive integers. Initially, each of the first $a$ bowls contains an apple, and each of the last $b$ bowls contains a pear.\\
		A legal move consists of moving an apple from bowl $i$ to bowl $i+1$ and a pear from bowl  to $j$ bowl $j-1$ , provided that the difference $i-j$ is even. We permit multiple fruits in the same bowl at the same time. The goal is to end up with the first $b$ bowls each containing a pear and the last $a$ bowls each containing an apple. Show that this is possible if and only if the product $ab$ is even.\\
		(译:有$a+b$个碗排成一行,编号为$1$至$a+b$,$a$\text{与}$b$为给定正整数.最初的时候,前$a$个碗各包含一个苹果,后$b$个碗各包含一个梨.一个合法的移动包含将一个苹果从碗$i$移动到碗$i+1$,并将一个梨从碗$j$移动到碗$i-1$,只要差$i-j$是偶数.我们的目标是:将最初的梨和苹果的顺序颠倒位置(即,前$b$个碗各包含一个梨,后$a$个碗各包含一个苹果.)证明这在且仅在乘积$ab$是偶数时是可能的.
        )\flushright{(Translator:Moyan Liang)}
		\label{pbl:ini}
    \end{problema}
    在这道题的证明中充分运用了"不变量"的思想,从而完成了本题的证明\footnote{值得一提的是:此题的完整解答须包含证明与构造可行的移动规则两个部分.笔者只完成了构造部分而被老师称为"本末倒置".}.
   
    今介绍一系列记法与定义,他们将会贯穿全文.
    \begin{enumerate}
        
        \item 我们将一个数集$S^{\prime}$为另一个数集$S$的排列记作$S^{\prime}=\mathbb{P}\left\{S\right\}.$
        \item 我们将一个数集$R^{\prime}$为另一个数集$R$中元素的组合记作$R^{\prime}=\mathbb{C}(\forall{e}\in R)$,或直接简单地记作$R^{\prime}$为另一个数集$R$中元素的组合记作$R^{\prime}=\mathbb{C}\left\{R\right\}.$
    \end{enumerate}

    本文共分三节,第一节介绍了一系列基础
        记法及定义,以及此文引题.第二节通过一系列问题引入"不变量"这一
        概念,同时介绍"不变量"的求解方法.第三节介绍"不变量"方法在现代
        数学竞赛中的应用与本人在研究此课题时的愚见.
    