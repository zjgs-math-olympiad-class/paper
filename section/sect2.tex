\section{"不变量"的求解方法}
下面给出引题的证明.
\begin{proof}[引题的证明]\footnote{此证明出自AOPS,由罗老师整理及翻译.}\footnote{这证明不包括构造过程,请参考本文附录.}
记奇数号碗中苹果和梨的总数分别为$A_1,P_1$,偶数号碗的苹果和梨的总数分别为$A_2,P_2$.
由于$(i-j)$时偶数,故每次操作都是将奇(偶)数号碗中的一个梨和苹果换到偶(奇)数号碗中,于是$(A_1-A_2)$
与$(P_1-P_2)$同时增加$2$或$-2$,这表明\begin{equation}
    M=(A_1-A_2)-(P_1-P_2)\label{eqn:inv}
\end{equation}
是本题合法操作的不变量.若$ab$为奇数,则$a,b$皆为奇数,初始状态下由\eqref{eqn:inv}
得$M=1-(-1)=2$,结束时$M=-1-1=-2$,由$M$为不变量推出矛盾,命题获证.
\end{proof}

此题的证明充分的利用了不变量的性质(就是说,它是不会变的),从而推出矛盾.事实上,在代数问题中
不变量亦是极其有用的.
\begin{problem}\footnote{出自:奥数教程(八年级),\S{8},例5.有改动.}
    已知三个数89,12,3,进行一种运算$Q$:
    \begin{equation}
        Q(a,b,c)=(\dfrac{a+b}{\sqrt{2}},\dfrac{a-b}{\sqrt{2}},c^2.)\label{eqn:qar}
    \end{equation}
    问:能否经过若干次运算$Q$,得到3个数90,14,10?证明它.
\end{problem}
此题初看没有头绪,而此类问题的一般解法是从\eqref{eqn:qar}中寻找不变量.
\begin{proof}
    \kaishu{答案显然是否定的.接下来给出证明.}
    \songti 我们知道
    \begin{equation}
        \left(\dfrac{a+b}{\sqrt{2}}\right)^2+\left(\dfrac{a-b}{\sqrt{2}}\right)^2+c^2=a^2+b^2+c^2.\label{eqn:eqv}
    \end{equation}
    \eqref{eqn:eqv}就是说,经过一次运算$Q$之前与之后的三个数的平方和是不变的.至此,我们找到了此题中的"不变量":平方和.
    而$${89}^2+12^2+3^2=8074,$$
    $$90^2+14^2+10^2=8396>8074.$$
    故经过题中题中规定的运算不能由已知的三个数得到$90,14,10$三个数.
\end{proof}
从上两例中我们可以看出"不变量"实际上是一种思想,要找到并证明它需要的是各种数学方法,如判别式,恒等变形等.下面分几类详细讨论.
\subsection{一般不变量}
\begin{problem}\footnote{选自罗老师提供的资料.}
    在某部落的语言中一共只有两个字母$A$与$B$,并且该语言具有以下性质:
    如果从单词中删去相邻的字母串$AB$,词义保持不变.或者说:单词中添加字母串$BA$
    或$AABB$,词义保持不变.

    问:$ABB=BAA$吗?
\end{problem}
\begin{solution}
    记每个单词为$(a,b)$,其中,$a$为该单词中$A$的个数,$b$为该单词中$B$的个数.
    现寻找"不变量".显然,在一串
\end{solution}